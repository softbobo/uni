% Robert Schulze
% Matrikelnummer: 555625
% thesis for the pro seminar on artificial intelligence at TUC 2019

% to do:
% check raw text, chapter by chapter
% insert text chapter by chapter
    % abstract
    % introduction
    % samplernn
    % dadabots and its results
    % discussion
% clear up citations of both carr/zukowski papers
% resolve all missing citations & references
% style front page
% make citations apalike

\documentclass[a4paper, 11pt]{report}

% include TUC logo
\usepackage{graphicx}
\usepackage{titling}
\author{Robert Schulze}
\title{Machine Music Generation: Using Neural Networks To Produce Metal Music}
\date{\today}


\usepackage{setspace}
\onehalfspacing

% set margins to 1 inch
\usepackage[margin=1in]{geometry}

% natbib for bibliography style
\usepackage{natbib}

% set font to TNR
\usepackage{mathptmx}
\usepackage[T1]{fontenc}

\begin{document}


\begin{titlepage}
    \begin{center}
        \includegraphics[height=5cm]{tuc-gruen.png}
        
        \begin{large}
            \thetitle \\
            \theauthor \\
            \date{\today}  
            
        \end{large}
        
    \end{center}
\end{titlepage}

% use roman numbering for chapters
\renewcommand{\thechapter}{\Roman{chapter}}

\setcounter{page}{1} 

\tableofcontents

\chapter*{Abstract}

This paper introduces the Dadabots project and uses it as an example for the 
current status of research on audio synthesis using Artificial Neural 
Networks. Throughout the text, I am going to explore the SampleRNN architecture 
and explain Recurrent Neural Networks in a slightly broader fashion, as these 
are the underlying technologies of Dadabots. In the next step I am going to 
explain the reasons of the project’s authors to choose these specific 
technologies. In the conclusion of this text I counter the notion of ‘the death 
of creativity’ through artificial intelligence as I hope to provide a provide 
a perspective on how Artificial Neural Networks and Deep Learning can 
provide new ways of artistic expression.

\chapter{Introduction}
Dadabots is an ongoing experiment by Zack Zukowsky and CJ Carr, with the main 
aim of exploring modern advances in AI (‘Artificial Intelligence’) research on 
music synthesis in conjunction with popular music, namely Black Metal. They 
claim, that most training data used in this area of research focuses on 
classical music which pursues the ideal of the most harmonic composition. 
In contrast, modern music composition increasingly uses timbre manipulations as 
creative techniques over explorations of different 
tonalities\cite{zukowski2018generating}, thus training an 
ANN (‘Artificial Neural Network’) on this kind of data may yield very different 
results compared to more traditional approaches. \\
Moreover, the authors state that most research in the realm of music synthesis 
produces output in the symbolic domain only\cite{zukowski2018generating}. In this context this typically means 
the generation of midi notes which represent melodies and rhythms. Opposed to 
this, Carr and Zukowsky’s approach produces music on a sample-based approach, 
one sample at a time\cite{mehri2016samplernn}\cite{zukowski2018generating}.  \\
For this purpose the researchers adapted an ANN, SampleRNN, and used different 
albums by bands of the likes of Meshuggah and NOFX as Datasets. The results of 
this process are publicly available on Bandcamp and in a 24/7 Youtube live 
stream and were first presented on the International Conference on Learning 
Representations (ICLR) in 2017. 

\chapter{SampleRNN}

\chapter{The Results of Dadabots}


\chapter{Discussion}


\bibliographystyle{plain}
\bibliography{biblio}


\end{document}