% Robert Schulze
% Matrikelnummer: 555625
% thesis for the pro seminar on artificial intelligence at TUC 2019

% to do:
% check raw text, chapter by chapter
% insert text chapter by chapter
    % abstract
    % introduction
    % samplernn
    % dadabots and its results
    % discussion
% resolve all missing citations & references
% style front page

\documentclass[a4paper, 11pt]{report}

% include TUC logo
\usepackage{graphicx}
\usepackage{titling}
\author{Robert Schulze}
\title{Machine Music Generation: Using Neural Networks To Produce Metal Music}
\date{\today}


\usepackage{setspace}
\onehalfspacing

% set margins to 1 inch
\usepackage[margin=1in]{geometry}

% natbib for bibliography style
\usepackage{natbib}

% set font to TNR
\usepackage{mathptmx}
\usepackage[T1]{fontenc}

\begin{document}


\begin{titlepage}
    \begin{center}
        \includegraphics[height=5cm]{tuc-gruen.png}
        
        \begin{large}
            \thetitle \\
            \theauthor \\
            \date{\today}  
            
        \end{large}
        
    \end{center}
\end{titlepage}

% use roman numbering for chapters
\renewcommand{\thechapter}{\Roman{chapter}}

\setcounter{page}{1} 

\tableofcontents

\chapter{Abstract}

This paper introduces the Dadabots project and uses it as an example for the 
current status of research on audio synthesis using Artificial Neural 
Networks. Throughout the text, I am going to explore the SampleRNN architecture 
and explain Recurrent Neural Networks in a slightly broader fashion, as these 
are the underlying technologies of Dadabots. In the next step I am going to 
explain the reasons of the project’s authors to choose these specific 
technologies. In the conclusion of this text I counter the notion of ‘the death 
of creativity’ through artificial intelligence as I hope to provide a provide 
a perspective on how Artificial Neural Networks and Deep Learning can 
provide new ways of artistic expression.


\chapter{Introduction}

\chapter{SampleRNN}

\chapter{The Results of Dadabots}


\chapter{Discussion}


\bibliographystyle{plain}
\bibliography{biblio}


\end{document}